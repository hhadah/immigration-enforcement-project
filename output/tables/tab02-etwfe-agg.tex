\begin{table}[H]
\centering\centering
\caption{Extended Two Way Fixed Effects: Aggregated Effects \label{tab:etwfe-agg}}
\centering
\resizebox{\ifdim\width>\linewidth\linewidth\else\width\fi}{!}{
\begin{threeparttable}
\begin{tabular}[t]{lccccc}
\toprule
  & Poor Health & School Lunch & Log School Lunch & Log SNAP & SNAP\\
\midrule
.Dtreat & \num{0}*** & \num{0.019}*** & \num{0.164}*** & \num{0.109}*** & \num{0.014}***\\
 & (\num{0}) & (\num{0.003}) & (\num{0.025}) & (\num{0.019}) & (\num{0.002})\\
\midrule
Observations & \num{976,648} & \num{976,648} & \num{976,648} & \num{976,648} & \num{976,648}\\
Cohort FE & X & X & X & X & X\\
Year FE & X & X & X & X & X\\
\bottomrule
\multicolumn{6}{l}{\rule{0pt}{1em}* p $<$ 0.1, ** p $<$ 0.05, *** p $<$ 0.01}\\
\end{tabular}
\begin{tablenotes}
\small
\item[1] \footnotesize{Each column is the results of the aggregat effects of extended two-way fixed effects estimation. 
                      Standard errors are clustered on the county level.}
\item[2] \footnotesize{The samples include first, second, third, and fourth+ generation Hispanic children ages 17 and below who live in intact families. 
                      A first-generation Hispanic child is one that is born in a Spanish-speaking country. 
                      A second-generation Hispanic child is one that is born in the United States with at least one parent born in a Spanish-speaking country.
                      Third-generation Hispanic immigrant children are native-born with native-born parents and at least one grandparent is born in a Spanish-speaking country.
                      country.
                      Fourth-generation+ are native born with native-born parents, all grandparents are born in the United States, and one parent self-reported Hispanic identity.}
\item[3] \footnotesize{Data source is the 1994-2019 Current Population Survey.}
\end{tablenotes}
\end{threeparttable}}
\end{table}
